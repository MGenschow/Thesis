\subsection{Manipulation}
While good inversion performance allows to accurately reconstruct a given image, the main goal of this work is to be able to consistently manipulate image attributes. Latent space manipulation is a research field closely connected to GAN inversion that aims at finding techniques to manipulate latent space representations of images such that the generated image from the manipulated latent space exhibits the desired edited attributes. Since the StyleGAN latent space is smoothly interpolable, moving in a carefully chosen direction in this space should also result in smooth interpolations in the desired semantic attributes \citep[p.6]{bermano2022state}. In general, semantic image manipulation using GANs can be performed using either specifically designed architectures based on conditional GANs or latent space manipulation techniques applied to pre-trained unconditional GANs. While models like StarGAN \citep{choi2018stargan} and StarGANv2 \citep{choi2020stargan} yield a more compact pipeline that does not require a specific model for each step, attributes have to be specified before training the generator. I chose to rely on an unconditional GAN model and additional latent space editing techniques since this approach is more general and yields better image quality \citep[p.2]{abdal2021styleflow}. Within those latent space navigation techniques, one can further differentiate between supervised (e.g. \cite{goetschalckx2019ganalyze}, \cite{shen2020interpreting}, \cite{wu2021stylespace}, \cite{yang2021discovering}) and unsupervised (e.g. \cite{ren2021learning}, \cite{yuksel2021latentclr}, \cite{harkonen2020ganspace}, \cite{shen2021closed}) approaches and between linear and non-linear (e.g. \cite{abdal2021styleflow}, \cite{li2023dystyle}, \cite{chen2022exploring}) approaches. The most important techniques will be briefly discussed below. \cite{harkonen2020ganspace} use an unsupervised approach by applying PCA on the latent space of a specific dataset. The principal components then correspond to specific attributes, however this annotation has to be done by manual inspection of resulting image manipulations. \cite{shen2021closed} proposed another unsupervised approach in which they use a semantic factorization method which is based on the principal components from the weights of the first layer of a pre-trained GAN. While unsupervised approaches may yield good results in the absence of labeled data, supervised approaches can leverage the information contained in the labels to find more accurate directions in the latent space. In one of the earlier works, \cite{goetschalckx2019ganalyze} propose a method to manipulate cognitive image properties like memorability. They achieve this by learning a transformer function that adds a learned direction to the initial latent code in $\mathcal{Z}$-space such that the assessor model at the end of the pipeline detects the altered memorability of the generated image. While this is a rather simple approach, it can be adapted to more complex workflows and other latent spaces. Building on the idea of learning a latent space offset based on assessor scores, \cite{shen2020interpreting} propose a technique that learns directions in latent space based on hyperplanes separating instances based on binary attributes. This method called InterFaceGAN will be explained in more detail in section \ref{sec:interfacegan}. While the above-mentioned methods mostly learn directions in latent space such that by traversing the latent space in this direction, the desired attribute changes in the output, non-linear approaches also yield promising results. StyleFlow, one of the most promising techniques, was proposed by \cite{abdal2021styleflow} and relies on normalizing flows that learn a mapping between $\mathcal{Z}$ and $\mathcal{W}$ conditioned on specific attributes. Manipulation is performed by reversing the mapping conditioned on the real attributes and then making a forward pass to the latent space using the manipulated attributes as the condition \citep[p.5]{abdal2021styleflow}. In this work, StyleFlow has been implemented and tested but was later disregarded due to its considerably higher computational cost and lower editing performance compared to InterFaceGAN. 
