\section{Discussion}
The goal of this work was the development of a fine-grained typicality manipulation method and the
exploration of various techniques to solve the various sub-tasks needed for this. While this goal and the desired main contribution have been achieved, a more detailed look at the implications of the results is necessary before concluding this work. Using two typicality measurements and two inversion techniques in the final experiments, the advantages and disadvantages of each combination of methods can be deducted when looking at the results presented in section \ref{sec:results}. Comparing the manipulation performance between DINOv2-based typicality scores and those based on disentangled embeddings reveals a clear trade-off between disentanglement and interpolability. Disentangled embeddings offer a rich, multi-dimensional understanding of typicality, while DINOv2-based embeddings are simpler, allowing only one-dimensional typicality manipulations. Attempts to disentangle manipulations from DINOv2-based typicality scores through conditioning were unsuccessful, which can be seen as a limitation of this work, although the overall typicality manipulations still produced satisfactory results. The advantage of the richer disentangled embeddings lies in the possibility of conditional, attribute-specific typicality manipulation, as shown with the sleeve length attribute conditioning. On the other hand, typicality manipulation based on disentangled embeddings does not always allow smooth and thus controlled interpolation. When traversing the latent space, sudden jumps in typicality scores and drastic changes in physical attributes can occur for some samples. To address this problem, a smoother distribution of typicality scores, on which latent space editing directions are based, is needed. This could be achieved by significantly expanding the dataset size. However, this was not possible during this work due to the limited number of articles available on "zalando.de". To artificially increase the dataset size, one could add data augmentations by manipulating images using the techniques described in this work and learning typicality scores based on the augmented dataset. This approach could be explored in future research to further improve typicality manipulation. For practitioners, this trade-off between disentanglement and interpolability is difficult to solve and the selection of the suitable approach often depends on the specific use-case. In high fashion, where a designer's creation is valued highly, often subtle changes in the design are desired. Here, using a simple typicality measure might be sufficient, since the changes applied can be much more subtle and the typicality scores change smoothly. However, for fast prototyping and automated design optimization, disentangled embeddings, that allow more control over the different attributes and that result in more drastic changes in the design are suitable.\\
Comparing the manipulation performance of the two tested inversion methods, a trade-off between reconstruction quality and editing performance becomes clear. While PTI is capable of reconstructing especially complicated samples much more realistically, its editing performance can degrade with the manipulation strength. The basic e4e inversion on the other hand lacks reconstruction quality but is capable of producing latent codes that are easily editable even into far regions of the latent space. Furthermore, PTI is significantly slower in producing inversions. Again, the selection of the correct approach depends on the specific use case and requirements a designer has for the design augmentation tool. For applications that require fast inference and strong manipulations, i.e. large steps in the latent space, e4e is the logical choice. However, if samples are difficult to invert or if the manipulation needs to be subtle, i.e. taking small steps in the latent space, using PTI is preferred. \\
Although the generated images from the GAN model and the inversion techniques are generally of high quality, artifacts can still occur in some generations. This is especially the case when the manipulation reaches ill-defined regions of the latent space. However, artifacts can also arise in any generation produced by the custom StyleGAN2-Ada model or any inversion technique. These quality issues highlight a weakness of using GANs compared to diffusion models, which typically produce images of higher quality. The main reason for using GAN-based techniques in this work was the well-behaved latent space of GANs.  However, with the growing popularity of diffusion models for image generation, new methods for image manipulation using these models are being published more frequently. During the duration of this thesis, several promising approaches for attribute manipulation with diffusion models were published. New techniques claim to have found methods to create a disentangled latent space for diffusion models similar to the StyleGAN latent space (e.g. \cite{dravid2024interpreting}). Exploring these techniques for typicality manipulation could be a promising direction for future research.