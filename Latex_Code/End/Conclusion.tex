\section{Conclusion}
This thesis investigated the manipulation of design typicality in fashion with generative deep neural networks, with a particular focus on the category of women's dresses. By leveraging the capabilities of the StyleGAN2-Ada model, the study aimed to adjust the typicality of fashion designs, a key aesthetic attribute linked to consumer preferences, while maintaining the overall integrity and coherence of the generated images. In this work, various techniques have been explored to achieve the sub-tasks of image generation, GAN inversion, typicality measurement, and latent space editing. The findings demonstrate that typicality can be effectively manipulated within the latent space of the StyleGAN2-Ada model, utilizing inversion techniques such as e4e and PTI. A supervised latent space manipulation method, InterFaceGAN, was successfully applied to achieve typicality editing. These results suggest that this approach could be a valuable tool for designers, enabling them to refine product designs in line with consumer expectations.
However, the study also highlighted several limitations and trade-offs. Notably, a trade-off between disentanglement and interpolability was observed in the two typicality measurements used. Additionally, a trade-off between inversion quality and editing performance was identified when comparing e4e and PTI inversions. In summary, this work contributes to the understanding of how generative models can be used to manipulate aesthetic attributes in fashion design. While further research is needed, the findings offer a foundation for future exploration in this area and potential applications in the design process.