\subsection{Typicality Measure}\label{sec:results_typicality_measure}
Looking at samples with the highest and lowest typicality scores for both typicality measures in figure \ref{fig:examples_dino_typicality} and \ref{fig:examples_disentangled_typicality}, it becomes clear that both measures are consistent. A typicality measure is consistent, if the samples that show a high (low) typicality are highly similar in their attributes while they are highly dissimilar to the sample with low (high) typicality scores. For the typicality measure based on DINOv2 embeddings, the highly typical samples have variation in color and sleeve lengths but the neckline and the overall shape of the dress are very similar. In the non-typical dresses, there are mostly very untypical dress shapes. In the typicality measure based on disentangled embeddings, the variation between the most typical dresses seems higher at first glance which is expected, since this is a simple combination of multiple disentangled attribute embeddings. One attribute that seems dominant however is the sleeve length, since the most typical dresses all show long sleeves while the least typical dresses are all sleeveless. If one excludes the attribute sleeve length, the most typical dresses show a waist-fitted shape as can be seen in figure \ref{fig:examples_disentangled_ex_sleevel_length} with varying sleeve lengths. Overall, the typicality measure based on the disentangled embeddings does not allow simple explanations like the one for the DINOv2-based measure, since the exact attributes, that determine that a dress is very typical can vary from dress to dress. 

