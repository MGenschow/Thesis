\section{Introduction}
The rapid development of generative neural networks for image generation has not only pushed the boundaries of computer vision research but also opened up new possibilities for productivity increases in many industries with drastic economic implications. McKinsey estimates, that only in product research and design, generative models could unlock \$60 Billion in productivity \citep{mc_kinsey_report}. Using generative models as assistants in the design process and thus automating parts of the value creation chain can hugely increase industrial productivity \cite[p.1]{regenwetter2022deep}. At the same time, many products are highly dependent on innovative or brand-specific designs, and especially in the fashion industry, the designer's experience and professional knowledge play an important role in the design process \cite[p.3]{sharma2021development}. However, generative models can augment these experts' design process and help them align their designs with the newest insights from consumer behavior research. In this work, one example of such a design augmentation for fashion designers is developed, focusing on the design typicality of fashion products. Research in consumer behavior has consistently demonstrated the significant role that typicality plays in product evaluation. Products are generally evaluated more positively, if they are more typical since typical products can be easier categorized by consumers (see \cite{barsalou1985ideals} p.1, \cite{sujan1985consumer} p.43) and processed more fluently (see e.g. \cite{veryzer1998influence} p.1, \cite{winkielman2006prototypes}, p.1). Additionally, more typical designs are more prevalent in consumers' everyday lives and can thus benefit from repeated exposure which can further increase consumer liking and familiarity \citep[p.23]{zajonc1968attitudinal}. In addition, consumers perceive atypical products as less reliable and functional than typical ones \cite[p.613]{schnurr2017impact}. Interestingly, the relationship between typicality and consumer liking is not linear. While more typical products can be categorized more easily which leads to higher consumer preferences, this only works to a certain extent. \cite{meyers1989schema} found that products which are moderately incongruent with the schema of their category (i.e. moderately atypical) are preferred over extremely incongruent products and completely congruent products. Products that are entirely congruent with the category schema are perceived as not noteworthy and exhibit lower consumer preference \cite[p.40]{meyers1989schema}. Furthermore, \cite{liu2017effects} found that moderate levels of typicality exhibit the highest consumer preferences with decreasing preferences for very high and very low typicality levels. \\
Keeping product typicality in mind is therefore central to the design process, as it directly influences consumer perceptions and decision-making processes. While expert human designers can focus on their individual design process, novel technologies from the domain of generative models can support optimizing a finished design regarding its design typicality for increased consumer preferences. Such a model could edit a design such that it is more (or less) typical than the initial draft by the designer, thus increasing consumer liking. Since the overall design of the fashion item should not change significantly, only minor and very targeted manipulations in the design are desired. This requires a very targeted and fine-grained image manipulation technique that can edit \textit{aesthetic} attributes, e.g. typicality, of the design in a disentangled manner (i.e. without altering other attributes). There are multiple works that research design augmentation for \textit{physical} attributes like sleeve length or color of fashion items (e.g. \cite{choi2023developing}, \cite{chen2020tailorgan}, \cite{ping2019fashion}, \cite{kwon2022tailor}). Furthermore, there is literature on \textit{aesthetic} attribute manipulation in other domains using generative neural networks. For example, \cite{goetschalckx2019ganalyze} manipulates images regarding visual attributes like memorability or emotional valence. To the best of the author's knowledge, a disentangled and targeted manipulation technique for aesthetic attributes like design typicality does not exist yet. Therefore, the goal of this work is the 
\begin{itemize}
    \item development of a fine-grained typicality manipulation method for aesthetic attributes and
    \item the exploration of generative models, GAN inversion methods, and latent space manipulation techniques to achieve this goal.
\end{itemize} 
The main contribution of this work is the development of a model that connects the research on aesthetic attribute manipulation using generative models and design augmentation in the fashion domain. While typicality is an \textit{aesthetic} attribute, manipulation of the design typicality may involve implicit manipulation of \textit{physical} attributes. The main difference to the works above, however, is the implicitness of \textit{physical} attribute manipulation, as the manipulation is targeted towards an \textit{aesthetic} attribute (typicality) which may result in manipulations of multiple \textit{physical} attributes of the fashion item.\\
To achieve the disentangled and targeted aesthetic attribute manipulation, a custom generative adversarial network (GAN), namely StyleGAN2-Ada \citep{stylegan2ada}, is trained for the garment category of women's dresses. Furthermore, multiple GAN-inversion models are tested and the best models are selected. To assess the typicality of a dress, two alternative typicality measurements are developed and in a final step, real dresses are manipulated using a latent space editing technique. The remainder of this work is organized as follows: Section 2 reviews the necessary preliminaries and relevant literature. Section 3 details the training data while section 4 explains the methodology, covering generation, inversion, typicality measurement, and latent space manipulation. Section 5 presents the results of each sub-method and the overall typicality manipulation. Section 6 discusses the method and results, and Section 7 concludes the paper.






 

