\clearpage

\begin{center}
    \textbf{Abstract}
\end{center}
The rapid evolution of generative adversarial networks (GANs) has significantly impacted various domains, including fashion, where design typicality is a critical factor in consumer preference. This study explores the disentangled manipulation of visual attributes, specifically design typicality, using generative models within the fashion industry. A comprehensive pipeline was developed utilizing a custom-trained StyleGAN2-Ada model and multiple GAN inversion techniques to perform fine-grained manipulations on dress designs. Two methods for measuring typicality were introduced: a simple approach using DINOv2 embeddings and a more complex method employing disentangled embeddings with a focus on specific fashion attributes. The results demonstrate the effectiveness of the proposed typicality manipulation techniques, both with and without conditioning on physical attributes like color and sleeve length. The implications of these techniques extend to enhancing design processes by optimizing fashion items for increased consumer appeal, bridging the gap between aesthetic attribute manipulation and practical design augmentation in the fashion domain.

\newpage